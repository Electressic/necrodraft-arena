\chapter{Quellen und Referenzen}
\label{chap:quellen}

\section{Game Design Referenzen}

\subsection{Gameplayinspiration}
\begin{itemize}
    \item \textbf{Teamfight Tactics (Riot Games):} Auto-Battler Mechaniken \\
          und strategische Positioning
    \item \textbf{Hearthstone Battlegrounds (Blizzard):} Ebenfalls Auto-Battler \\
          Mechaniken mit Fokus auf dem Kartensystem
    \item \textbf{Slay the Spire (Mega Crit):} Deckbuilding Progression
    \item \textbf{Divinity: Original Sin 2 (Larian):} Modulare Equipmentsets \\
          und Fähigkeitenkombinationen
\end{itemize}

\subsection{UI/UX Design}
\begin{itemize}
    \item \textbf{Auto Chess Genre:} Drag \& Drop Interface
    \item \textbf{Path of Exile (Grinding Gear Games):} Komplexe \\
          Inventorymanagement
    \item \textbf{Monster Hunter (Capcom):} Equipmentslot basierte \\
          Progression
\end{itemize}

\section{Technische Ressourcen}

\subsection{Unity Engine}
\begin{itemize}
    \item \textbf{Unity Technologies:} Unity 6 LTS Engine und \\
          Universal Render Pipeline
    \item \textbf{Unity Documentation:} Official Unity Scripting \\
          API Reference
    \item \textbf{Unity Learn Platform:} ScriptableObject Architecture \\
          Tutorials
\end{itemize}

\subsection{Asset Herkunft und Eigenleistung}

\subsubsection{Eigenleistung}
\begin{itemize}
    \item \textbf{Code:} Vollständige C\# Skripte und Unity Implementierung in Eigenarbeit
    \item \textbf{Game Design:} Spielmechaniken, Balancing und Systemarchitektur
    \item \textbf{UI/UX Design:} Interfacelayout und Benutzerführung
    \item \textbf{Art Assets:} Alle visuellen Elemente außer explizit aufgeführten Ausnahmen
\end{itemize}

\subsubsection{AI-Generierte Inhalte}
\begin{itemize}
    \item \textbf{Hintergrundbilder:} AssemblyBackground (2), Background und \\
          GameplayBackground - erstellt mit Imagen 4 von AI Studio
    \item \textbf{Klassenbilder:} Alle Sprites im Ordner sprites/classes - \\
          erstellt mit Imagen 4 von AI Studio
\end{itemize}

\subsubsection{Externe Ressourcen}
\begin{itemize}
    \item \textbf{Fonts:} TextMeshPro Standardschriftarten (Unity)
    \item \textbf{Hintergrundmusik:} von \\
          \url{https://pixabay.com/users/clavier-music-16027823/}\\
          \url{?utm_source=link-attribution&utm_medium=referral}\\
          \url{&utm_campaign=music&utm_content=354468}
    \item \textbf{Button-Soundeffekte:} von \\
          \url{https://ateliermagicae.itch.io/be-not-afraid-uimenu-sfx}\\
          \url{?download}
\end{itemize}

\section{Code Bibliotheken und Tools}

\subsection{Unity Packages}
\begin{itemize}
    \item Universal Render Pipeline \\
          (com.unity.render-pipelines.universal) - Version 17.1.0
    \item 2D Animation (com.unity.2d.animation) - Version 10.2.0  
    \item Input System (com.unity.inputsystem) - Version 1.14.0
    \item Visual Studio Editor \\
          (com.unity.ide.visualstudio) - Version 2.0.23
\end{itemize}

\subsection{Development Tools}
\begin{itemize}
    \item \textbf{Visual Studio Code:} Primäre IDE für C\# \\
          Development
    \item \textbf{Git:} Versionskontrolle und Source Management
    \item \textbf{Unity Hub:} Projekt- und Engine-Management
\end{itemize}

\section{Externe Entwicklerressourcen}

\subsection{Community und Documentation}
\begin{itemize}
    \item \textbf{Unity Forums:} Community Support für technische \\
          Probleme
    \item \textbf{Stack Overflow:} C\# und Unity-spezifische \\
          Programmierungsfragen
    \item \textbf{GitHub:} Open Source Unity-Projekte als Referenz
\end{itemize}
