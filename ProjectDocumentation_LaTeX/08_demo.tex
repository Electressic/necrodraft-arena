\chapter{Demo und Zusatzmaterialien}
\label{chap:demo}

\section{Projekt-Repository}

\subsection{Source Code}
\begin{itemize}
    \item \textbf{Repository:} necrodraft-arena
    \item \textbf{Hauptentwicklung:} Unity Projekt mit vollständigem Source Code
    \item \textbf{Dokumentation:} LaTeXbasierte Projektdokumentation
\end{itemize}

\subsection{Projekt-Struktur}
\begin{itemize}
    \item \texttt{Assets/Scripts/} - Gesamte C\# Codebase
    \item \texttt{Assets/ScriptableObjects/} - Part- und Class-Definitionen  
    \item \texttt{Assets/Scenes/} - Spielszenen \\
          (MainMenu, ClassSelection, MinionAssembly, Gameplay)
    \item \texttt{ProjectDocumentation\_LaTeX/} - Diese Dokumentation
\end{itemize}

\section{Gameplay-Features}

\subsection{Implementierte Core-Features}
\begin{itemize}
    \item $\checkmark$ Vollständiges Part-System mit Fähigkeiten
    \item $\checkmark$ Drag \& Drop UI für intuitive Partassembly
    \item $\checkmark$ Automatisierte Combatsimulation
    \item $\checkmark$ Wellenprogression mit Minionunlocks
    \item $\checkmark$ Save/Load System für Spielfortschritt
    \item $\checkmark$ Mergesystem für Partupgrades
    \item $\checkmark$ Multiklassensystem mit verschiedenen Spezialisierungen
\end{itemize}

\subsection{Demoumfang}
\begin{itemize}
    \item \textbf{Spielbare Waves:} 1-20 mit progressiver Schwierigkeit
    \item \textbf{Part-Varietät:} 3 Themes (Skeleton, Ghost, Zombie) mit jeweils 4 Slot-Typen \\
          (Allerdings nur Skeleton Parts implementiert)
    \item \textbf{Progression:} Minionunlocks und Raritysteigerung über Wellenfortschritt
\end{itemize}

\section{Technische Demonstration}

Im Repository befindet sich ein Ordner technische Demonstration, der eine ausführliche 
technische Präsentation der Kernfeatures enthält. Diese umfasst:
\begin{itemize}
    \item \textbf{Gameplay-Demo:} Kurzes Gameplay-Video, das die Kernmechaniken zeigt
    \item \textbf{UI-Demo:} Screenshots der Benutzeroberfläche und deren Interaktionen
\end{itemize}

\section{Weiterentwicklung}

\subsection{Geplante Features}
\begin{itemize}
    \item Erweiterte Visual Effects und Animationen
    \item Zusätzliche Partthemes und Implementierung der Klassen
    \item Balancingverbesserungen basierend auf Gameplaytests
    \item Verbesserung des Nutzererlebnisses durch UI-Optimierungen
\end{itemize}

\subsection{Potentielle Erweiterungen}
\begin{itemize}
    \item Multiplayermodi für kompetitives Gameplay
    \item Moddingsupport durch ScriptableObjectsystem
    \item Mobile Platform Portierung
    \item Narrative Kampagne
\end{itemize}
