\chapter{Gameplay}
\label{chap:gameplay}

\section{Core Gameplay Loop}

NecroDraft Arena folgt einem zyklischen Gameplay-Loop mit zwei Hauptphasen:

\begin{enumerate}
    \item \textbf{Assembly Phase} (1-3 Minuten): Strategische Optimierung der Minion-Armee
    \item \textbf{Combat Phase} (30-90 Sekunden): Automatisierter Kampf zwischen den Armeen
\end{enumerate}

\subsection{Assembly Phase}
Nach jedem Kampf erhält der Spieler automatisch 3 zufällige Parts ins Inventar. In dieser Phase können Parts:
\begin{itemize}
    \item An Minions in deren 4 Ausrüstungsslots (Head/Torso/Arms/Legs) ausgerüstet werden
    \item Miteinander zu stärkeren Varianten gemergt werden (3 identische Parts → 1 höhere Seltenheit)
    \item Aus dem Inventar gelöscht werden, um Platz zu schaffen
\end{itemize}

\subsection{Combat Phase}
Vollständig automatisierte Kämpfe basieren auf:
\begin{itemize}
    \item \textbf{Einfache KI-Regeln:} Units greifen den Gegner ganz links an, ausgenommen sind Gegnertypen die unterschiedliche Positionen angreifen.
    \item \textbf{Ability-Triggern:} Parts aktivieren spezielle Fähigkeiten basierend auf Kampfereignissen
    \item \textbf{Siegbedingung:} Die letzte überlebende Armee gewinnt die Runde
\end{itemize}

\section{Part-System}

\subsection{Ausrüstungsslots}
Jeder Minion verfügt über 4 Ausrüstungsslots mit thematischen Schwerpunkten:
\begin{itemize}
    \item \textbf{Head:} Präzisions- und mentale Fähigkeiten (Kritische Chance, Rüstungsdurchdringung, Kritischer Schaden)
    \item \textbf{Torso:} Defensive und Ausdauer-Fähigkeiten (Rüstung, Gesundheit, Angriff)
    \item \textbf{Arms:} Offensive Fähigkeiten (Angriff, Kritische Chance, Rüsterungsdurchdringung)
    \item \textbf{Legs:} Bewegungs-Fähigkeiten (Ausweichen, Rüstung, Gesundheit)
\end{itemize}

\subsection{Special Abilities}
Parts können spezielle Fähigkeiten basierend auf ihrer Kategorie besitzen:

\subsubsection{Guardian Abilities}
\begin{itemize}
    \item \textbf{Taunt:} Zwingt Gegner, diesen Minion anzugreifen
    \item \textbf{Shield Wall:} Blockiert Angriffe auf Verbündete hinter diesem Minion
    \item \textbf{Damage Sharing:} Teilt einen Prozentsatz des Schadens mit benachbarten Verbündeten
\end{itemize}

\subsubsection{Assault Abilities}
\begin{itemize}
    \item \textbf{Flanking:} Bonusschaden wenn von Randpositionen angreifend
    \item \textbf{Focus Fire:} Erhöhter Schaden gegen Gegner unter 50\% HP
    \item \textbf{Momentum:} Steigender Angriff pro Runde (zurückgesetzt bei Schaden)
\end{itemize}

\subsubsection{Marksman Abilities}
\begin{itemize}
    \item \textbf{Range Attack:} Kann hintere Reihe von hinten angreifen
    \item \textbf{Overwatch:} Chance auf Konterangriff bei Verbündeten-Angriffen
    \item \textbf{Hunter:} Massiver Bonusschaden gegen Gegner unter 25\% HP
\end{itemize}

\subsubsection{Support Abilities}
\begin{itemize}
    \item \textbf{Healing:} Heilt verwundete Verbündete pro Runde
    \item \textbf{Inspiration:} Gibt benachbarten Verbündeten permanenten Angriffsbonus
    \item \textbf{Battle Cry:} Einmaliger kampfweiter Angriffsbuff für alle Verbündeten
\end{itemize}

\subsubsection{Trickster Abilities}
\begin{itemize}
    \item \textbf{Mobility:} Kann Positionen mit Verbündeten tauschen
    \item \textbf{Phase Step:} Chance, Schaden durch Positionstausch zu vermeiden
    \item \textbf{Confuse:} Lässt Gegner zufällige Ziele angreifen
\end{itemize}

\subsection{Thematische Spezialisierung}

\textbf{Bone Theme} (Aggressiv, präzise):
\begin{itemize}
    \item Hohe Affinität: Assault (60\%) und Marksman (30\%) Abilities
    \item Schwerpunkt: Angriff, Kritische Treffer, Rüstungsdurchdringung
\end{itemize}

\textbf{Flesh Theme} (Defensiv, ausdauernd):
\begin{itemize}
    \item Hohe Affinität: Guardian (60\%) und Support (30\%) Abilities  
    \item Schwerpunkt: Gesundheit, Verteidigung, Teamunterstützung
\end{itemize}

\textbf{Spirit Theme} (Vielseitig, unberechenbar):
\begin{itemize}
    \item Hohe Affinität: Trickster (40\%) und ausgewogene Verteilung
    \item Schwerpunkt: Utility-Fähigkeiten, flexible Stat-Verteilung
\end{itemize}

\section{Steuerung}

\subsection{Maussteuerung}
\begin{itemize}
    \item \textbf{Drag \& Drop:} Parts zwischen Inventar und Minionslots bewegen
    \item \textbf{Rechtsklick:} Partdetails und Mergeoptionen anzeigen
    \item \textbf{Linksklick:} Menüs bestätigen, Kampf starten
\end{itemize}

\section{Strategische Tiefe}

\subsection{Build-Archetypen}
\begin{itemize}
    \item \textbf{Guardian Build:} Taunt + Shield Wall für maximalen Schutz
    \item \textbf{Assault Build:} Flanking + Focus Fire für hohen Schaden
    \item \textbf{Marksman Build:} Range Attack + Hunter für Fernkampfdominanz
    \item \textbf{Support Build:} Healing + Inspiration für Teamunterstützung
    \item \textbf{Trickster Build:} Mobility + Phase Step für Unberechenbarkeit
\end{itemize}

\subsection{Strategische Entscheidungen}
\begin{itemize}
    \item \textbf{Commitment vs. Flexibilität:} Slots für Macht opfern oder Vielseitigkeit bewahren
    \item \textbf{Part-Verteilung:} Einen Superminion oder ausgewogenes Team erstellen
    \item \textbf{Seltenheits-Abwägung:} Epic Part nehmen oder auf Synergie warten
    \item \textbf{Counter-Strategien:} Gegnerische Builds antizipieren und kontern
\end{itemize}
