\chapter{Weitere Inhalte}
\label{chap:weitere_inhalte}

\section{Thematischer Rahmen}

NecroDraft Arena ist im Dark Fantasy Setting angesiedelt, in dem der Spieler als Nekromant auftritt. 
Das Spiel verzichtet bewusst auf eine tiefe Narrative und konzentriert sich stattdessen auf mechanische Tiefe und strategisches Gameplay.

\subsection{Setting}
\begin{itemize}
    \item \textbf{Genre:} Dark Fantasy mit strategischen Elementen
    \item \textbf{Protagonisten:} Verschiedene Nekromantenklassen mit unterschiedlichen Spezialisierungen
    \item \textbf{Zielkonflikt:} Arena basierte Kämpfe zwischen Untotenarmeen
\end{itemize}

\section{Visueller Stil und Farbschema}

\subsection{Designprinzipien}
\begin{itemize}
    \item \textbf{Top-Down Perspektive:} Klassische strategische Übersicht
    \item \textbf{Modularer Stil:} Körperteile sind visuell unterscheidbar und kombinierbar
\end{itemize}

\subsection{Farbpalette}
\begin{itemize}
    \item \textbf{Primärfarben:} Dunkle Grautöne und Blautöne für UI-Elemente
    \item \textbf{Akzentfarben:} Verschiedene Farben für Part-Seltenheiten (Common, Rare, Epic)
    \item \textbf{Thematische Farben:} Knochenweiß für Skeleton-Theme, Ätherisch-Blau für Ghost-Theme
\end{itemize}

\section{Audio-System}

Das Spiel nutzt ein zentralisiertes Audio-System mit MusicManager und SettingsManager.

\subsection{Komponenten}
\begin{itemize}
    \item \textbf{Hintergrundmusik:} Durchgehende Musik mit Fadeübergängen
    \item \textbf{UI-Sounds:} Button-Clicks und Menü-Navigation
    \item \textbf{Kampf-Audio:} Angriffs- und Schadenssounds (geplant)
\end{itemize}

\subsection{Audio-Einstellungen}
\begin{itemize}
    \item Separate Lautstärkeregelung für Master, Musik und Soundeffekte
    \item Persistente Speicherung der Audioeinstellungen
\end{itemize}

\section{KI Verhalten}

\subsection{Combat KI}
Das Kampfsystem basiert auf positionsabhängigen Targeting-Regeln mit 6 verschiedenen Gegnertypen:

\begin{itemize}
    \item \textbf{Bruiser Enemies:} Greifen immer den Minion ganz links in der vorderen Reihe an
    \item \textbf{Archer Enemies:} Zielen prioritär auf die hintere Reihe (weiteste Position)
    \item \textbf{Assassin Enemies:} Attackieren den Minion mit der niedrigsten aktuellen HP (beliebige Position)
    \item \textbf{Sniper Enemies:} Fokussieren den Minion mit dem höchsten Angriffswert
    \item \textbf{Bomber Enemies:} Verursachen Flächenschaden auf alle benachbarten Positionen
    \item \textbf{Guardian Enemies:} Blockieren Angriffe und schützen andere Gegner
\end{itemize}

\subsection{Positionssystem}
\begin{itemize}
    \item \textbf{Formation:} 2 Reihen mit je 3 Positionen (vorne/hinten)
    \item \textbf{Vordere Reihe:} Erhält mehr Schaden, blockiert Zugang zur hinteren Reihe
    \item \textbf{Hintere Reihe:} Geschützt, aber begrenzte Angriffsmöglichkeiten
    \item \textbf{Strategische Platzierung:} Spieler muss Formation an Gegnertypen anpassen
\end{itemize}

\subsection{Balance Mechaniken}
\begin{itemize}
    \item Vorhersagbare Targeting-Muster ermöglichen strategische Planung
    \item Deterministische Kämpfe für optimale Partoptimierung
    \item Skalierung der Gegnerstärke basierend auf Wellenfortschritt
\end{itemize}
