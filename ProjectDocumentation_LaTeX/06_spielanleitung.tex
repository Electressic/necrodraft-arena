\chapter{Spielanleitung}
\label{chap:spielanleitung}

\section{Einleitung}

Willkommen bei NecroDraft Arena! Als Nekromant erschaffst du mächtige Untoten-Armeen durch das strategische Kombinieren von Körperteilen. 
Jeder Teil verleiht deinen Minions einzigartige Fähigkeiten, die durch zusammenfügend immer stärker werden.
In automatisierten Kämpfen bewähren sich die strategischen Entscheidungen der Aufbauphasen.

\section{Spielstart}

\subsection{Erste Schritte}
\begin{enumerate}
    \item \textbf{Klassen-Auswahl:} Wähle deine Nekromantenklasse für verschiedene Spezialisierungen (momentan nur Bone Weaver verfügbar)
    \item \textbf{Erstes Minion:} Du startest mit einem Basisminion und einigen Startteilen
    \item \textbf{Assembly Phase:} Übersicht deines Minions, deines Inventars und der verfügbaren Parts
    \item \textbf{Combat Phase:} Positioniere dein Minion an einer der 6 Positionen in der Arena und beobachte den automatisierten Kampf
\end{enumerate}

\section{Steuerung}

\subsection{Maussteuerung}
\begin{itemize}
    \item \textbf{Drag \& Drop:} Parts aus dem Inventar auf Minionslots ziehen
    \item \textbf{Hover:} Tooltips für detaillierte Partinformationen
    \item \textbf{Buttons:} Navigation zwischen Minions und Modi (bspw. zusammenfügen der Parts oder ausrüsten der Parts an einem Minion)
\end{itemize}

\subsection{UI Navigation}
\begin{itemize}
    \item \textbf{Pfeiltasten:} (wenn vorhanden) Zwischen Minions wechseln
    \item \textbf{Inventar:} Verfügbare Parts am rechten Teil des Bildschirms.
    \item \textbf{Part-Slots:} Head, Torso, Arms, Legs für jeden Minion
\end{itemize}

\section{Kernmechaniken}

\subsection{Partsystem}
\begin{itemize}
    \item \textbf{4 Slot Typen:} Jeder Part passt nur in seinen spezifischen Slot
    \item \textbf{Seltenheiten:} Common (grau), Uncommon (grün), Rare (blau), Epic (violett)
    \item \textbf{Themes:} Bone (aggressiv), Flesh (defensiv), Spirit (vielseitig)
\end{itemize}

\subsection{Special Abilities}
\begin{itemize}
    \item \textbf{Kategorien:} Guardian, Assault, Marksman, Support, Trickster
    \item \textbf{Aktivierung:} Parts haben individuelle Fähigkeiten basierend auf ihrem Budget
    \item \textbf{Beispiele:} Taunt (zwingt Gegner zum Angriff), Range Attack (Fernkampf), Healing (heilt Verbündete)
\end{itemize}

\subsection{Partmanagement}
\begin{itemize}
    \item \textbf{Merging:} 3 identische Parts → 1 höhere Seltenheit
    \item \textbf{Löschen:} Unerwünschte Parts entfernen für Platz
    \item \textbf{Strategische Planung:} Potentiell Special Abilities über Rohstats priorisieren
\end{itemize}

\section{Progression}

\subsection{Wellensystem}
\begin{itemize}
    \item \textbf{Welle 1-7:} Akt I - Grundlagen lernen (max. 3 Minions)
    \item \textbf{Welle 8-14:} Akt II - Erweiterte Strategien (max. 4 Minions)
    \item \textbf{Welle 15-20:} Akt III - Endgame Content (max. 5 Minions)
\end{itemize}

\subsection{Minionunlocks}
\begin{itemize}
    \item Welle 5: 2. Minionslot
    \item Welle 9: 3. Minionslot
    \item Welle 13: 4. Minionslot
    \item Welle 17: 5. Minionslot
\end{itemize}

\section{Strategische Tipps}

\subsection{Für Anfänger}
\begin{itemize}
    \item \textbf{Abilities nutzen:} Parts mit Special Abilities sind oft stärker als reine Statparts
    \item \textbf{Themefokus:} Spezialisiere Minions auf ein Theme für bessere Fähigkeitensynergien
    \item \textbf{Slotplanung:} Behalte Parts für strategische Builds und Merging Möglichkeiten
\end{itemize}

\subsection{Fortgeschrittene Taktiken}
\begin{itemize}
    \item \textbf{Partbudgeting:} Rare und Epic Parts strategisch verteilen
    \item \textbf{Komposition der Armee:} Guardian, Assault, Support Rollen ausbalancieren
    \item \textbf{Mergetiming:} Parts upgraden vs. Fähigkeitendiversität abwägen
\end{itemize}

\section{Häufige Fehler}

\begin{itemize}
    \item \textbf{Fähigkeiten ignorieren:} Einzelne hohe Stats ohne spezielle Fähigkeiten sind oft schwächer
    \item \textbf{Ungenutzte Parts:} Inventar nicht voll ausnutzen
    \item \textbf{Mischen der Themes:} Verschiedene Themes pro Minion verwässern Synergien
    \item \textbf{Zu frühes Mergen:} Parts mergen bevor strategische Builds etabliert sind
\end{itemize}
