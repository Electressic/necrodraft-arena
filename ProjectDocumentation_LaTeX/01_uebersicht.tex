\chapter{Übersicht}\label{chap:uebersicht}

\section{Spielname und Genre}

\textbf{NecroDraft Arena} ist ein strategischer Top-Down Autobattler mit Deck-Building-Elementen. 
Das Spiel kombiniert taktische Positionierung mit modularem Minion-Crafting in einem Dark Fantasy Setting.

\section{Kurzbeschreibung}

Als Nekromant erschaffen Spieler einzigartige Untoten-Armeen durch das Zusammensetzen modularer Körperteile. 
Jedes Teil verleiht spezielle Fähigkeiten, die kombiniert noch stärker werden.
In automatisierten Kämpfen bewähren sich die strategischen Entscheidungen der Aufbauphasen.

\section{Zielgruppe}

\begin{itemize}
    \item 	\textbf{Primär:} Strategie-Enthusiasten (18-35 Jahre) mit Affinität zu Deck-Building und Autobattler-Spielen
    \item 	\textbf{Sekundär:} Spieler mit Interesse an Dark Fantasy und taktischen Kämpfen
    \item 	\textbf{Erfahrungslevel:} Mittlere bis hohe Gaming-Erfahrung
\end{itemize}

\section{Plattform}

\begin{itemize}
    \item 	\textbf{Zielplattform:} PC (Windows 10/11, 64-bit)
    \item 	\textbf{Eingabegeräte:} Maus und Tastatur
\end{itemize}

\section{Kern-Gameplay Loop}

Der Spielzyklus besteht aus drei aufeinanderfolgenden Phasen:

\begin{enumerate}
    \item \textbf{Combat Phase} (30-90 Sek): Automatisierte Kämpfe zwischen Minion-Armeen
    \item \textbf{Assembly Phase} (2-4 Min): Ausrüstung, Merge und Verwaltung der Parts  
    \item \textbf{Results Phase} (10 Sek): Sieg/Niederlage Feedback und Part-Belohnungen
\end{enumerate}

Nach jedem Kampf erhält der Spieler automatisch 3 zufällige Parts, die direkt ins Inventar wandern. 
In der Assembly-Phase können Parts an Minions ausgerüstet, miteinander zu stärkeren Varianten 
gemergt oder gelöscht werden - alles in derselben Szene.

\section{Part-System Übersicht}

\subsection{Part-Aufbau}
Jeder Part verfügt über ein dynamisches Punktebudget, das auf Stats verteilt wird:

\begin{itemize}
    \item \textbf{Common Parts:} 12 Punkte Budget (1-2 Stats)
    \item \textbf{Uncommon Parts:} 20 Punkte Budget (2 Stats)
    \item \textbf{Rare Parts:} 32 Punkte Budget (2-3 Stats)
    \item \textbf{Epic Parts:} 50 Punkte Budget (3-4 Stats)
\end{itemize}

\subsection{Thematische Spezialisierung}
Parts gehören zu einem von drei Untoten-Themen mit distinkten Stat-Affinitäten:

\begin{itemize}
    \item \textbf{Skeleton:} Geschwindigkeit, Kritische Treffer, Reichweite (Glass Cannon)
    \item \textbf{Zombie:} Gesundheit, Verteidigung, Ausdauer (Tank)
    \item \textbf{Ghost:} Ausgewogene Stats, Vielseitigkeit (Hybrid)
\end{itemize}

\subsection{Merge-System}
Drei identische Parts (gleicher Slot und Seltenheit) können zu einem stärkeren Part der nächsten Seltenheitsstufe verschmolzen werden.

\section{Alleinstellungsmerkmale}

\begin{itemize}
    \item \textbf{Modulares Part-System:} Vollständig anpassbare Einheiten durch 4-Slot Ausrüstungssystem (Head/Torso/Arms/Legs)
    \item \textbf{Dynamische Part-Generierung:} Prozedural generierte Parts mit themenbasierten Stat-Affinitäten statt fester Items
    \item \textbf{Advanced Merge-Mechanik:} 3 gleiche Parts verschmelzen zu einem stärkeren Part der nächsten Seltenheitsstufe
    \item \textbf{Budget-basierte Stats:} Jeder Part hat ein Punktebudget, das dynamisch auf 1-4 Stats verteilt wird
    \item \textbf{Thematische Spezialisierung:} Skeleton (Glass Cannon), Zombie (Tank), Ghost (Hybrid) mit distinkten Stat-Profilen
    \item \textbf{Integrated Assembly:} Part-Management und Minion-Ausrüstung in einer einzigen, nahtlosen Szene
\end{itemize}
