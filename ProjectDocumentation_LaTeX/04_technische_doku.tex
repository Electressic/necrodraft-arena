\chapter{Technische Dokumentation}
\label{chap:technische_doku}

\section{Game Engine und verwendete Tools}

\subsection{Unity Engine}
\begin{itemize}
    \item \textbf{Version:} Unity 6+ mit 2D Core Template
    \item \textbf{Render Pipeline:} Universal Render Pipeline (URP) 17.1.0
    \item \textbf{Input System:} New Unity Input System 1.14.0
    \item \textbf{Zielplattform:} PC Standalone (Windows 64-bit)
\end{itemize}

\subsection{Externe Pakete}
\begin{itemize}
    \item 2D Animation (10.2.0) für Sprite-Animationen
    \item Visual Studio Editor (2.0.23) für Code-Integration
\end{itemize}

\section{Code Architektur}

\subsection{Zentrale Managersysteme}
Das Projekt folgt einem Manager-Pattern mit statischen Klassen für globale Systeme:

\begin{itemize}
    \item \textbf{GameData:} Verwaltet Spielfortschritt und Wellenstatus
    \item \textbf{MinionManager:} Zentrale Minionroster Verwaltung
    \item \textbf{PlayerInventory:} Part Inventar mit Eventsystem
    \item \textbf{SaveSystem:} Serialisierung und Persistierung
\end{itemize}

\subsection{ScriptableObject Architektur}
Datenklassen sind als ScriptableObjects implementiert für Design-Flexibilität:

\begin{itemize}
    \item \textbf{PartData:} Modulare Körperteil Definitionen
    \item \textbf{NecromancerClass:} Spielerklassen mit Boni
    \item \textbf{MinionData:} Basisminion Eigenschaften
    \item \textbf{EnemyData:} Gegnerdefinitionen
\end{itemize}

\subsection{UI Architektur}
Szenenbasierte UI Verwaltung mit Managerkomponenten:

\begin{itemize}
    \item \textbf{MinionAssemblyManager:} Partausrüstung und Minionverwaltung
    \item \textbf{MainMenuManager:} Hauptmenü mit Save/Load Integration
    \item \textbf{SettingsManager:} Audio- und Grafikeinstellungen
    \item \textbf{ClassSelectionManager:} Klassenauswahl mit Preview
\end{itemize}

\section{Part System Implementation}

\subsection{Stat Budget System}
Parts nutzen ein punktebasiertes Stat-System:

\begin{itemize}
    \item \textbf{Stat-Kosten:} HP (1 Punkt), Attack (1 Punkt), Defense (2 Punkte), Kritische Chance (2 Punkte), Kritischer Schaden (1 Punkt), Rüstungsdurchdringung (3 Punkte) 
    \item \textbf{Rarity Budget:} Common (8 Punkte), Uncommon (14 Punkte), Rare (22 Punkte), Epic (35 Punkte)
    \item \textbf{Special Abilities:} Variable Kosten je nach Stärke (3-20 Punkte)
    \item \textbf{Dynamische Generation:} Themenbasierte Statverteilung und Ability-Zuordnung
\end{itemize}

\subsection{Special Abilities Kosten}
\begin{itemize}
    \item \textbf{Guardian:} Taunt (4/6/8), Shield Wall (6/9/12), Damage Sharing (5/8/11)
    \item \textbf{Assault:} Flanking (4/7/10), Focus Fire (5/8/11), Momentum (6/9/12)
    \item \textbf{Marksman:} Range Attack (8/12/16), Overwatch (7/11/15), Hunter (5/8/11)
    \item \textbf{Support:} Healing (3/5/7), Inspiration (4/6/8), Battle Cry (8/12/16)
    \item \textbf{Trickster:} Mobility (10/15/20), Phase Step (12/18/24), Confuse (6/9/12)
\end{itemize}

\section{Build und Deployment}

\subsection{Buildkonfiguration}
\begin{itemize}
    \item \textbf{Plattform:} PC Standalone (Windows x64)
    \item \textbf{Kompression:} LZ4 für Balance zwischen Größe und Ladezeit
    \item \textbf{Development Build:} Für Debug-Features in Prototyping
\end{itemize}

\subsection{Assetmanagement}
\begin{itemize}
    \item Resources-Ordner für dynamisch geladene ScriptableObjects
    \item Addressables System für zukünftige Modularität
    \item Texturekompression für optimale Dateigröße
\end{itemize}
