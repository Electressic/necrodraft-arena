\chapter{Entwicklungsprozess}
\label{chap:entwicklungsprozess}

\section{Projektplanung und Meilensteine}

\subsection{Entwicklungsansatz}
Das Projekt folgte einem iterativen Prototyping-Ansatz mit Fokus auf Core-Mechaniken:

\begin{enumerate}
    \item \textbf{Phase 1:} Grundlegende Part System Implementierung
    \item \textbf{Phase 2:} UI System und Minion Assembly Interface
    \item \textbf{Phase 3:} Combatsystem und WWellenprogression
    \item \textbf{Phase 4:} Savesystem und Gamebalance
\end{enumerate}

\subsection{Technologie Entscheidungen}
\begin{itemize}
    \item \textbf{Unity 6:} Für moderne 2D-Features und URP-Integration
    \item \textbf{ScriptableObjects:} Für datengetriebenes Design ohne Codeänderungen
    \item \textbf{Eventsystem:} Für lose gekoppelte Manager Kommunikation
\end{itemize}

\section{Herausforderungen und Lösungsansätze}

\subsection{Partsystem Komplexität}
\textbf{Problem:} Ursprünglich prozentbasierte Stats waren schwer zu balancieren und zu verstehen.

\textbf{Lösung:} Migration zu punktebasiertem Budget-System mit klaren Kosten pro Statpunkt. 
Zuerst waren auch Setboni geplant, durch das Mergesystem wurde dies doch gestrichen und durch ein Fähigkeitensystem ersetzt. 
Die Fähigkeiten funktionieren auch über das punktebasiertem Budget-System, was die Balance vereinfacht.

\subsection{UI Skalierung}
\textbf{Problem:} Inkonsistente UI Darstellung auf verschiedenen Auflösungen. Elemente wie Buttons und Tooltips waren nicht skalierbar. 


\textbf{Lösung:} SceneConsistencyManager für automatische Canvas-Skalierung und Resolution-Management.
Außerdem wurde die generelle Auflösung auf Full-HD (1920x1080) festgelegt, um das Skalieren der UI zu vereinfachen. 
Pixelbasierte UI-Elemente wurden mit verschiedenen Skalierungseinstellungen versehen (z.B. keine Kompression und kein Filter).

\subsection{Savesystem Integration}
\textbf{Problem:} Komplexe Objektreferenzen zwischen Minions, Parts und ScriptableObjects. Parts wurden nicht im Inventar gespeichert, sondern nur in der Scene. 
Die Darstellung der Parts im Inventar war nicht persistent, was zu Inkonsistenzen führte.

\textbf{Lösung:} Serialisierungsklassen mit Stringbasierten Referenzen statt direkter Objektlinks.

\section{Designentscheidungen}

\subsection{Automatisierte Kämpfe}
\textbf{Entscheidung:} Vollständig automatisierte Combatphase ohne Spielereingriff.

\textbf{Begründung:} Fokus auf strategische Partoptimierung statt Mikromanagement im Kampf.

\subsection{Single Scene Assembly}
\textbf{Entscheidung:} Alle Part Management Features in einer Szene kombiniert.

\textbf{Begründung:} Reduziert Szenenwechsel und ermöglicht dem Spieler alle Features (bspw. Mergen der Parts oder Verwalten der Minions) auf einen Blick zu nutzen.

\subsection{Managerpattern}
\textbf{Entscheidung:} Zentrale statische Manager für globale Spielzustände.

\textbf{Begründung:} Vereinfacht Datenzugriff zwischen Szenen und reduziert Singletonkomplexität.

\section{Erreichte Ziele und Weiterentwicklung}

\subsection{Erfolgreich implementiert}
\begin{itemize}
    \item Vollständig funktionsfähiges Partsystem mit speziellen Fähigkeiten
    \item Intuitive Drag \& Drop UI für Part Assembly
    \item Persistente Save/Load Funktionalität
    \item Skalierbare Wellenprogression mit Freischaltung von Minions
    \item Modulare Codebase für einfache Erweiterungen
\end{itemize}

\subsection{Verbesserungspotential}
\begin{itemize}
    \item \textbf{Performance:} Object Pooling für UI Elemente
    \item \textbf{Content:} Weitere Klassen, mehr Partthemes, Ausreifung des Gameplayflows durch interaktive Entscheidungen nach jeder Welle (z.B. Pfadfindung)
    \item \textbf{Polish:} Animationen und Visual Effects
    \item \textbf{Balance:} Erweiterte Gameplaytests für Stattuning und Fähigkeitenbalancing
\end{itemize}

\subsection{Fazit}
NecroDraft Arena demonstriert erfolgreich die Kernmechaniken eines modularen Deck-Building-Autobattlers. 
Das fundierte technische Framework ermöglicht einfache Contenterweiterungen und bietet eine solide Basis für die Weiterentwicklung zum vollständigen Spiel.
