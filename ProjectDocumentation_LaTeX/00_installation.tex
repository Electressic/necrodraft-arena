\chapter{Installation und Setup}
\label{chap:installation}

\section{Systemanforderungen}

\begin{itemize}
    \item \textbf{OS:} Windows 10 (64-bit) oder höher
    \item \textbf{RAM:} 4 GB (empfohlen: 8 GB)
    \item \textbf{Speicherplatz:} 150 MB
    \item \textbf{Eingabe:} Maus und Tastatur
\end{itemize}

\section{Installation für Entwickler}

\subsection{Voraussetzungen}
\begin{enumerate}
    \item \textbf{Unity Hub:} Download von \url{https://unity.com/download}
    \item \textbf{Unity Editor:} Version 6000.1.4f1
    \item \textbf{Git:} Optional für Versionskontrolle
\end{enumerate}

\subsection{Projekt Setup}
\begin{enumerate}
    \item Projektarchiv entpacken oder Repository klonen
    \item Unity Hub öffnen → Add project from disk
    \item \texttt{necrodraft-arena} Hauptordner auswählen
    \item Erste Öffnung: Unity importiert Assets und Pakete (ca. 2-3 Minuten)
\end{enumerate}

\subsection{Dependencies}
Das Projekt nutzt folgende Unity Packages:
\begin{itemize}
    \item Universal Render Pipeline (17.1.0)
    \item 2D Animation (10.2.0)
    \item Input System (1.14.0)
    \item Visual Studio Editor (2.0.23)
\end{itemize}

\section{Installation für Spieler}

\subsection{Windows Standalone}
\begin{enumerate}
    \item Release-Version herunterladen
    \item ZIP-Archiv entpacken
    \item \texttt{NecroDraftArena.exe} ausführen
\end{enumerate}

\section{Erste Schritte}

\subsection{Für Entwickler}
\begin{itemize}
    \item \textbf{Projekt-Struktur:} \texttt{Assets/Scenes/} (Spielszenen), \texttt{Assets/Scripts/} (C\# Code), \texttt{Assets/ScriptableObjects/} (Part-Daten)
    \item \textbf{Test:} MainMenu.unity öffnen → Play-Button → Spiel testen
    \item \textbf{Build:} File → Build Settings → PC Standalone → Build and Run
\end{itemize}

\subsection{Für Spieler}
\begin{itemize}
    \item \textbf{Steuerung:} Maus für UI-Navigation, Drag \& Drop für Part-Assembly
    \item \textbf{Spielstart:} Necromancer-Klasse wählen → Parts sammeln → Combat starten
\end{itemize}
